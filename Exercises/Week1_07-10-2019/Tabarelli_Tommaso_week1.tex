\documentclass[11pt, a4paper]{report}
\usepackage{mathptmx}
\usepackage[T1]{fontenc}
\usepackage[utf8x]{inputenc}
\usepackage[english]{babel}
\usepackage{graphicx}
\usepackage{float}
\usepackage{amsmath}
\usepackage{mathtools}
\usepackage{amsfonts}
\usepackage{enumerate}
\usepackage[caption = false]{subfig}

\usepackage[margin=1.5cm]{geometry}
\usepackage[font={small,it}]{caption}


\begin{document}

\begin{titlepage}

\begin{center}
\LARGE{Università degli Studi di Padova}\\
\line(1,0){450}\\
\begin{figure}[H]
\centering

\end{figure}
\vspace{-0.2em}
\Large{Dipartimento di Fisica e Astronomia ‘‘Galileo Galilei’’}\\
\vspace{2em}
\Large{Corso di Laurea in Physics of Data}\\
\vspace{8em}
\Huge{\textsc{\textbf{Quantum information and computation: week 1 exercises}}}\\
\vspace{2em}
\LARGE{\textsc{of}}\\
\vspace{1em}
\huge{\textsc{Tommaso Tabarelli}}\\
\vspace{11em}
\LARGE{Anno accademico 2019-2020}\\
\line(1,0){450}
\end{center}

\end{titlepage}




\section*{Exercise 1}
The first exercise is a program that write in terminal "Hello world".

\section*{Exercise 2}
In the second exercise, after writing the first version of the program using 2 bytes to represent integers (int*2), the compiler can recognize that there is an "out of range" error due to the numbers used and it does not compile. Newer compiler versions suggest the following:
\textit{Error: Arithmetic overflow converting INTEGER(4) to INTEGER(2) at (1). This check can be disabled with the option ‘-fno-range-check’}.\\
Using that flag the compiler act without warnings but the executable program gives a \textit{nosense} result (as expected).\\ \\
In the version that uses 4 bytes to represent integers (int*4) there are no problems (with the used numbers).

\section*{Exercise 3}



\end{document}